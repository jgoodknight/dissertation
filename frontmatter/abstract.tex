% the abstract
%% An abstract, not to exceed 350 words

Biological and artificial light-harvesting systems as well as ion-trap quantum computing all have physics which are highly dependent on electron-nuclear interactions, so it is incredibly important to be precise in how we treat these interactions.  People very commonly, however, invoke the approximation that molecules do not absorb or emit light differently, no matter what the state is of their nuclear geometry is; put more technically, that the molecular dipoles do not change with respect to nuclear coordinate.  The first person to make this approximation was E.U. Condon (in 1928), and so it is named after him.  Now, however, that state-of-the-art spectroscopy has short-enough pulses to isolate nuclear motion, this approximation is beginning to fail in its relevancy to modern investigations of the interactions between electronic and nuclear degrees of freedom.

I start with an example of how a variation in the transition dipole moment can completely change the interpretation of an entire class of experiment by confusing one physical effect for another when investigating the mechanism of photosynthetic energy transfer.  Recognizing that having a non-Condon transition dipole moment can be a problem, we investigate a method for estimating the functional form of the transition dipole as a function of the nuclear coordinate through a linear electronic absorption spectroscopy experiment.  Then, we look at how a variation in the transition dipole moment actually causes a molecule to heat during the operation of an ultrafast laser spectroscopy and how that heating would manifest as an isotropic infrared signal.  Then, finally, we look at how relaxing the Condon approximation both heating up and causes loss of fidelity in a trapped-ion Molmer-Sorenson gate: a ubiquitous architecture in the field of trapped ion quantum computing.
