% the acknowledgments section

It seems standard to have a sappy acknowledgments section in every thesis so this will be no exception.  Many tears went into writing this section so please read it.

My grad school story starts after orientation when a kind first-year graduate student who had been a Harvard undergrad offers me a tour of campus beyond the department.  Jacob Sanders' initial kindness turned into a great friendship that got us through many boring seminars, teaching together and an exciting trip 20 hour to Paris after a 3 day layover at a conference on Quantum Effects in Biology.

My co-hort had many other special people whose constant friendship helped me get through grad school and the particularly tough first year.  Especially Ryan Babbush, Amy Gilson, Katherine Phillips and Denise Alfonso.

The classes that came after mine weren't so bad either, actually, and I'm particularly grateful for the stoic realism of Thomas Markovich and the free hugs of Jennifer Wei; they were both great friends in Grad School.

Then of course, there are the awesome older grad students to whom I owe a great deal.  Jarrod McClean is by far the most intellectually impressive person I have ever met and I had some incredibly eye-opening conversations with him as I was starting out in the research group.  Roberto Olivares was also so kind as to take me on during my rotation in the research group even though I'm reasonably confident I actively slowed him down.  Joel Yuen, who started basically all of the work that my PhD was on, also helped me a great deal.  Jon Welch was also a constant friend and colleague who I owe a lot to.

I also owe a tremendous debt to many of the fantastic postdocs that have been in the group while I've been here.  Starting with the most important: Jacob Krich.  My PhD would probably not happened if he hadn't offered to mentor me as if he was still a postdoc from his new faculty job in Ottawa, as I was lost and adrift as a third year student.  I will forever be grateful to him.

I also want to mention Johannes Hachman, John Parkhill, David Tempel, Man Hong Yung, Xavier Andrade, Joonsuk Huh, Dmitrij Rappoport, and Sule Atahan-Evrenk who all taught me little bits here and there and who I all owe a debt of gratitude for making grad school a much more interesting experience.

\begin{center}
   ***
\end{center}

Outside of the lab I have many people to thank!  Starting close to the lab, I have to thank the former department administrator Allen Aloise for helping me through some difficult and more bureaucratic parts of grad school.

Stepping a bit further away from lab, I always greatly enjoyed my teaching obligations, and Gregg Tucci was always an incredible resource for teaching and navigating the department.  But I also have to particularly thank Pia Sorenson for hiring me to teach the most awesome course possibly in Harvard's History: ``Science and the Physical Universe 27: Science and Cooking'' not once but twice.  Still sorry I couldn't have taught it even more times, but finishing this document took some precedence, alas.


I also must admit to spending a lot of time outside of lab engaged in the most fulfilling music and theatrical performance opportunities I have ever had--and likely will ever have.

Each show with the Harvard-Radcliffe Gilbert and Sullivan players was an absolute joy and gave me reasons to laugh and to sing and the people were amazing to work with; there are far too many to mention them all.

Then, 4 years in the Harvard University Choir gave me the constant feeling that I was a terrible musician but I loved every minute of it; I particularly want to thank Laszlo Seress for being such a good friend and literally dragging me to auditions, and also Edward Elwyn Jones (Gund University Choirmaster and Organist in the Memorial Church at Harvard University: a Space of Grace\textregistered in Harvard Yard) who was an inspiriationally patient leader, wonderful parent and outstanding musician.

Working as a resident tutor in Pforzheimer House has also been an immense joy and honor.  That would never have happened without Emily Stokes-Rees, Anne Harrington and John Durant.  And it would have been a lot more boring without Gabe Katsch and David Francis!  I have learned from all of my students in the house, and there are once again too many to mention.

Then there was the UC Berkeley Alumni Club of New England which I got myself involved with when I went with my good friend Michael Chang, to a football watching party near Fenway and eventually ended up leading it after my other good friend Chad Smith left for California.  I too left this last year, and want to thank Geoffrey Liou for helping me not worry; I know the group is in great hands!

The more science I learned, the more I realized the whole Universe is mind-boggingly complex, wonderful, and mysterious.  Contrary to what one might expect, I think I came out of grad school more Catholic than I started, and I want to thank Fr. George Salzmann (who is also a scientist with a PhD in biochemistry) for being such a great friend and intellectual conversationalist these 4 years.  I never thought a friendly but unexpected lunch invitation after mass one Sunday would turn into such a long friendship.

\begin{center}
   ***
\end{center}

Coming back to science, I have to thank Keith Nelson and Dudley Herschebach for being on my Graduate Advising Committee.  I chose them both, as much for their reputation as scientists, as for their reputations as decent human beings.  That being said, the handful of chances I got to present my research and ideas to them, I was constantly blown away by their scientific insight and probing questions.  One of my biggest regrets in grad school is not meeting with these amazing scientists more often.  I also owe a tremendous debt of gratitude to Rick Heller for stepping in to be on my dissertation defense committee.

\begin{center}
   ***
\end{center}

My family has been incredibly important to me throughout this whole experience.  My Dad (Greg Goodknight) in particular has always been there to talk with me when I needed and been very patient about not seeing me as much as he did when I was in college in California.  I know my dear mother (Teri Cahill) would be just as helpful had we not lost her over 15 years ago to cancer.  I wish more than anything she was here to read this acknowledgement in person, but I have faith she may now have other means of reading it now.  My amazing step-mother (Rosie Stephenson-Goodknight) has been there for me and so supportive.  As have her two sons, my amazing stepbrothers Sean and John who were always both so excited to hear me describe my research at Christmas and Thanksgiving.

\begin{center}
   ***
\end{center}

Then there is the person who has made the last year+ of grad school the best year of my life despite being stressed out of my mind.  Tess Brooks: the love of my life has made my life full in a way I never knew before.  If you're reading this at some point in the future because some dude named ``Joseph Knightbrook'' claimed he wrote it; he's not lying.  He just got married to the most amazing woman in the world and changed his name from what is on the front.  I can't wait to spend the rest of my life with her.


\begin{center}
   ***
\end{center}


Then there's the man who's second in line behind me for fault on this PhD: my advisor Al\'an Aspuru-Guzik.  Many graduate students have to go through graduate school constantly wondering if their bosses actually care about them as a person as opposed to caring about them as a citation-producing machine, but I never had that problem.  Al\'an cultivates the most friendly and healthy atmosphere in his lab and never says no to a crazy new idea.  I thank my advisor Al\'an from the bottom of my heart for taking a chance on a kid from Berkeley with no theory experience at all and for helping me through a wild ride.
