Spectroscopy exploits the electromagnetic field to interrogate the properties of matter.  If laser pulses are employed, in general, one can gain more information about matter the more laser pulses involved. Therefore, one of the main goals of the field has been the detection of signals involving many light-matter interaction events.  Spencer et. al. report in Nature Communications \cite{GAMERS} on the newest form of laser spectroscopy on the block: GAMERS (gradient-assisted multidimensional electronic Raman spectroscopy) and its application to several dye molecules.  GAMERS uses 6 pulses to measure perhaps the most complex multidimensional spectroscopic signals yet obtained. GAMERS allows for the possibility of obtaining  intricate detail on the coupling of electronic states to nuclear vibrational motion.  Through this newly-gained insight into electron-nuclear coupling, GAMERS is poised to provide new information related to quantum dynamics involving  perovskite solar cells\cite{Perovskite}, singlet fission materials\cite{singletFissionVibration}, quantum nanostructures\cite{NanostructureVibrations}, and photosynthestic complexes\cite{FMO1}. All of these systems exhibit details of the electronic-vibrational interactions that require further exploration and that have been the subject of several discussions in the literature.
