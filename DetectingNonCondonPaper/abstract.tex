The Condon approximation assumes that transition dipole moments do not vary with nuclear coordinates. It is often employed in physical chemistry and spectroscopy settings without fully verifying if the approximation is justified.   Recent scholarship shows both that such variation non-trivially affects spectroscopic observables and that many systems do indeed exhibit significant variation in their electronic transition dipole moments.  The most striking recent example is the work of Heller and co-workers~\cite{hellerGraphene} where non-varying transition dipole moments were essential to understand the spectra of polyacetylene and graphene.  Thus a better understanding of how such phenomenon affects even basic spectroscopic variables would be useful, as would a tool to measure the extent to which a system has a variation in its transition dipole moment.  We discuss how polynomial variation with respect to the nuclear coordinate in the electronic transition dipole moment affects the interpretation of the electronic absorption spectrum and use the result to investigate the transition dipole structure of tetracene.
