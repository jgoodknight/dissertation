Frequency-integrated pump-probe experiments with ultrafast pump and probe have been previously proposed to provide a witness for electronic coherence. That is, given a coherent oscillation of interest, such an experiment can distinguish whether the coherence has electronic character or is solely a vibrational effect. That proposal assumed the Condon approximation, that the transition dipole moments are independent of the nuclear coordinates.
We consider the effects of non-Condon transition dipoles on the witness protocol and show that for systems with small Huang-Rhys factor, the witness protocol will give false positives even with very small non-Condon effects. For larger Huang-Rhys factors, we show that a modified interpretation of the pump probe data continues to give a functioning witness protocol, up to a limiting strength of non-Condonicity, which we quantify.
