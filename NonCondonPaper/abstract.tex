A pump-probe spectroscopy experiment done on a system without electronic coherence, in the impulsive limit will exhibit oscillations which die down as the pulse width approaches zero, but only in the case that it has no variation in its transition dipole with respect to nuclear coordinate.  Variations in the transition dipole, once widely approximated to be constant with respect to nuclear coordinate, are not always so as recent scholarship demonstrates.  We show that even minimal, linear variations in the transition dipole break the ability to diagnose electronic coherences in the previously suggested manner.  Our results should give pause to anyone working in dynamic spectroscopy and in the interest of rigor, future calculations should at least test the effect of the Condon approximation before making it.
