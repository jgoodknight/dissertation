A frequency-integrated pump-probe spectroscopy experiment done on a system with vibrational structure, but without an electronic coherence, will exhibit oscillations which die down as the pulse width approaches zero, contrary to the case where there is an electronic coherence and the oscillations will increase.  This neat up/down behavior is only, however, in the case that the system in question has no variation in its transition dipole with respect to nuclear coordinate (the Condon approcimation).  We consider the effects of linear variations in the transition dipole and show how for small vibrational diplacements, the previously suggested protocol gives false-positives for the existance of electronic coherence.  For larger vibrational displacements, we show that a modified interpretation of the pump probe data continues to give a functioning protocol with fewer false positives, up to a limiting strength of the slope of the transition dipole, which we quantify
