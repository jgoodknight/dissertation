\section{Introduction}
One normally surmises the presence of electronic coherence through the identification of oscillations in pump probe or 2D Electronic Spectroscopy but vibrational structure can also cause oscillations in those kinds of spectra.  Yuen-Zhou, et al  in 2011 proposed a novel way of isolating electronic coherences in a system which also had vibrational structure: a pump probe spectrum performed with an impulsive laser pulse (or one sufficiently close to impulsive) will exhibit no oscillations if the system only has a vibrational coherence but will have oscillations if there is an electronic coherence.  Thus this experiment, when performed would act as a witness--a yes or a no--for the existence of electronic coherence in a system which also have vibrational structure.

This experiment, however, has a few problems.  Primarily, the fact that perfectly impulsive laser pulses are, as yet, unattainable with current equipment.  While we wait for laser pulses to get to infinite energy, the original propsoal by Yuen-Zhou suggested performing a series of pump probe experiments at increasingly narrow pulse widths.  Johnson et al\cite{allanWitness} showed that this procedure worked well: when the amplitude of a studied system's pump-probe oscillations increase, approaching a non-zero value as the laser pulses used approach impulsive, it implies the existence of electronic coherence in the system.  If, however, the amplitudes go to zero as the laser pulses get closer to impulsive, then a the system only contains vibrational coherences..

While Johnson may have proved that the experiment Yuen-Zhou proposed will work with laser pulses that modern experimental equipment can actually produce, there has not yet been a treatment of the effect of variations in the transition dipole with respect to nuclear coordinate.  Such a variation would effectively mean that the strength of the molecule as an emitter and absorber would vary as the molecule's vibrational state varies which could induce oscillations in dyanmical spectra such as pump probe, transient grating or 2D electronic.

What effect, then, would a basic variation in the transition dipole have on a pump-probe experiment and, more to the point, will a variation such as this cause an oscillation in a pump probe spectra taken at the impulsive limit?  We take it upon ourselves to investigate.

\section{System Setup}
To begin the investigation, we exclusively look at a system which only has vibrational coherence: a monomer with one excited state and with only one virational coordinate.   Since the goal is to be able to isolate vibrational effects and get rid of them, this presents the easiest test for false positives: if this system presents oscillations in a spectrum, then we know where they come from.

All of the systems we consider will thus look like so: we construct a vibrational monomer with a ground electronic state $\ket{g}$ and an excited electronic state $\ket{e}$ with a different harmonic vibration in both the ground and excited state such that the Hamiltonian looks like this:
\begin{align}
	H_0 &=  \sum_n \hbar \omega_{\gamma}  \left(n + \frac{1}{2} \right)  \ket{n_{\gamma}}\ket{g}\bra{g} \bra{n_{\gamma}} \\
   &+ \sum_m \left(  \hbar \omega_{\epsilon}  \left(m + \frac{1}{2} \right) + \omega_e \right)  \ket{m_{\epsilon}} \ket{e}\bra{e} \bra{m_{\epsilon}}
\end{align}
With the greek indeces corresponding to the vibrational states and the roman indeces corresponding to the electronic states.  We say that the minimum of the two potential energy wells are separated by a distance $\delta x$.

Now we give this system a transition dipole moment so it can interact with input laser fields.  For simplicity's sake we ignore the vector nature of both the transition dipole and the laser pulses, instead preferring to effectively collapse them all into one direction.
\begin{align}
	\hat{\mu}(x) &= \mu (x)  \left( \ket{e}\bra{g} + \ket{g} \bra{e} \right)
\end{align}
where $\mu (x)$ is where the transition dipole can get interesting and break the Condon approximation.  For now we leave it general


\section{Setting up the Experiment}
We concern ourselves exclusively with Pump Probe spectroscopy using gaussian laser pulses in this paper which differ only the their central time.  Which means all of our pulses will have the form:
\begin{align}
	E_{\text{pu}} &= \frac{E_0}{\sqrt{2 \pi \sigma^2}} e^{-\frac{t^2}{2 \sigma^2} } \left[ -e^{i \omega_c t} + e^{i \omega_c t} \right]\\
	E_{\text{pr}} &= \frac{E_0}{\sqrt{2 \pi \sigma^2}} e^{-\frac{\left(t-T\right)^2}{2 \sigma^2} } \left[ -e^{i \omega_c \left(t-T\right)} + e^{i \omega_c \left(t-T\right)} \right]
\end{align}
We chose the normalization scheme the same way Johnson et al did: to keep the integral of the absolute value of the electric field constant across all possible pulse widths.  We can further split the laser pulses into positive frequency excitation pulses:
\begin{align}
	E_{\text{pu}+}(t) &= \frac{E_0}{\sqrt{2 \pi \sigma^2}} e^{-\frac{t^2}{2 \sigma^2} } e^{-i \omega_c t} \\
	E_{\text{pr}+}(t) &= \frac{E_0}{\sqrt{2 \pi \sigma^2}} e^{-\frac{\left(t-T\right)^2}{2 \sigma^2} }  e^{-i \omega_c \left(t-T\right)}
\end{align}
and negative frequency relaxation pulses:
\begin{align}
	E_{\text{pu}-}(t) &= \frac{E_0}{\sqrt{2 \pi \sigma^2}} e^{-\frac{t^2}{2 \sigma^2} } e^{i \omega_c t} \\
	E_{\text{pr}-}(t) &= \frac{E_0}{\sqrt{2 \pi \sigma^2}} e^{-\frac{\left(t-T\right)^2}{2 \sigma^2} }  e^{i \omega_c \left(t-T\right)}
\end{align}
We then define an interaction of the beam $Q$ as being the integral:
\begin{align}
	\ket{\psi_{Q} (t)}  &= -\frac{i}{\hbar} \int_{-\infty}^{t} U(t, \tau) E_{P}(\tau) \hat{\mu}(x) \ket{\psi_0 (\tau)} d \tau
\end{align}
and where we would define a second interaction as:
\begin{align}
	\ket{\psi_{Q, P} (t)}  &= -\frac{i}{\hbar} \int_{-\infty}^{t} U(t, \tau) E_P (\tau) \hat{\mu}(x) \ket{\psi_Q (\tau)} d \tau
\end{align}
From here we can define all of the relevant electric field emissions which obey phase matching and the rotating wave approximation and so contribute to the pump probe signal:
\begin{align}
	E_{GSB1} (t) &=  i \bra{\psi_{0} (t)} \hat{\mu} (x) \ket{\psi_{\text{pu+, pu-, pr+}} (t)}\\
	E_{ESA} (t) &=  i \bra{\psi_{\text{pu+}} (t)} \hat{\mu} (x) \ket{\psi_{\text{pu+, pr+}} (t)}\\
	E_{GSB2} (t) &=  i \bra{\psi_{\text{pu+, pr-}} (t)} \hat{\mu} (x) \ket{\psi_{\text{pu+}} (t)}\\
	E_{SE} (t) &=  i \bra{\psi_{\text{pu+, pu-}} (t)} \hat{\mu} (x) \ket{\psi_{\text{pr+}} (t)} \\
	S_{i} (T) &= 2 \text{Re} \left[ \int E^*_{\text{pr+}} (t) E_i (t) dt  \right]
\end{align}



\section{Unused}

\begin{align}
	\hat{\mu}(x) &= \mu (x)  \left( \ket{e}\bra{g} + \ket{g} \bra{e} \right)\\
	\mu (x) &= \mu_0 \left[ 1 + \lambda \frac{x}{\delta x}  \right]
\end{align}
where $\lambda$ is a convenient dimensionless parameter which quantifies the amount of variation in transition dipole that a ground state wavepacket would experience in the excited state.  Now if we derive the expressions for the pump-probe using impulsive pump and probe beams:
\begin{align}
	E_{\text{pu}} &= E_0 \delta(t) \\
	E_{\text{pr}} &= E_0 \delta(t - T)
\end{align}
and an initial state $\ket{\Psi (t=0)} = \ket{\eta_{\gamma}} \ket{g}$ then the signal from the two ground state bleach pathways will be
\begin{align}
	S_{\text{GSB}} &=  2 E_0^4 \sum_{k}   \cos \left[  \omega_{\gamma}  \left(   k - \eta \right) T \right]  \left(  \bra{\eta_{\gamma}}  \mu^2 (x) \ket{k_{\gamma}} \right)^2
\end{align}
and the stimulated emission signal will be
\begin{align}
	S_{\text{SE}} &= E_0^4 \sum_{j, l}   \cos  \left[\omega_{\epsilon}  \left(   j - l \right) T \right]   \bra{\eta_{\gamma}}  \mu (x) \ket{j_{\epsilon}} \bra{j_{\epsilon}}  \mu^2 (x) \ket{l_{\epsilon}}\\
   &\times \bra{l_{\epsilon}}  \mu (x) \ket{\eta_{\gamma}}
\end{align}
If $\lambda=0$, the overlap collapses to Kroenicker delta function and both the stimulated emission and ground state bleach completely lose all time-dependence--the major result of .  If, however $\lambda \neq 0$, then both can have oscillations.  To quickly demonstrate this in the ground state bleach signal, we transform the nuclear part of the transition dipole to use the raising and lowering operators of the ground state:
\begin{align}
	\mu (x) &= \mu_0 \left[ d_{\gamma} + c_{\gamma} \left(  a^{\dagger}_{\gamma} + a_{\gamma}\right) \right] \\
	\mu^2(x) &=  \mu_0^2 \left[  d_{\gamma}^2 + 2 d_{\gamma} c_{\gamma} \left( a_{\gamma}^{\dagger} + a_{\gamma}  \right)  + c_{\gamma}^2 \left( a_{\gamma}^{\dagger} a_{\gamma}^{\dagger}  + a_{\gamma}a_{\gamma} + 1 + 2\hat{N}_{\gamma}  \right)\right]
\end{align}
which then turns the dipole overlap part of the ground state bleach term into
\begin{align}
	  \bra{\eta_{\gamma}}  \mu^2 (x) \ket{k_{\gamma}} &= \delta_{\eta,k} (d^2_{\gamma} + c_{\gamma}^2 + 2kc_{\gamma}^2) + 2c_{\gamma} d_{\gamma} \left( \sqrt{k+1} \delta_{\eta, k+1} + \sqrt{k}\delta_{\eta, k-1} \right)  +c_{\gamma}^2 \left( \sqrt{k+1} \sqrt{k+2} \delta_{\eta, k+2} + \sqrt{k}\sqrt{k-1}\delta_{\eta, k-2} \right)
\end{align}
where a nonzero $\lambda$ has now induced a ground state bleach oscillation in both the fundamental and second harmonic frequency of the ground state because the terms proportional to $\delta_{\eta, k + a}$ will induce an oscillation in the signal at the $a$th harmonic.
