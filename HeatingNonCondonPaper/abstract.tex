The Condon approximation that off-diagonal electronic transition dipole moments do not vary with nuclear coordinates, is not as solid an approximation in the world of ultrafast, dynamic spectroscopies as it was in the days of gas-phase rovibrational spectroscopy when Condon first made the approximation.  Recent scholarship shows both that such variation non-trivially effects spectroscopic observables and that many systems do indeed exhibit variation in their transition dipole moments.  Thus a better understanding of how such phenomenon affects even basic spectroscopic variables would be useful, as would a tool to measure the extent to which a system has a variation in its transition dipole moment.  We discuss the effects of a simple, linear transition dipole moment variation on the vibrational state of the molecule during laser excitation and predict that for an IR-active vibrational mode, one would see isotropic signal at the fundamental vibrational frequency if and only if there is a non-trivial transition dipole moment in the impulsive pulse limit.  We the look at this IR effect for real pulses and conclude that it would not a useful discriminator between Condon and non-Condon moments.
